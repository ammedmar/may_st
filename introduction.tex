
\section{Introduction}

The role Steenrod operations play in Stable Homotopy Theory is hard to overstate. The reason is that, given the representability of the cohomology functor, these operations together with the Bockstein homomorphism can be used to give a complete description of the algebraic structure naturally present on the mop-$p$ \mbox{cohomology} algebra of spaces. For the even prime, Steenrod squares were introduced in \cite{steenrod47products} via an explicit choice of coherent homotopical corrections to the broken symmetry of Alexander-Whitney's chain approximation to the diagonal. Later, for odd primes,	their definition was given non-effectively using arguments based on the mod-$p$ homology of symmetric groups \cites{steenrod53symmetric, steenrod53cyclic, steenrod62operations}. This viewpoint enhanced the conceptual understanding of the operations and allowed for many advances \cites{adem1952iteration, milnor1958dual, adams1995stable}, but lacked the concreteness of their definition at~$2$.

In recent years, thanks to the development of new applications of cohomology ---most notably in Applied Topology and Quantum Field Theory--- the need to have a definition of Steenrod operations that can be effectively computed in specific examples has gain considerable importance. In Applied Topology, the use of Persistence Homology \cites{carlsson2009data, edelsbrunner2008persistent} has open many new interdisciplinary research directions \cites{de2007coverage,chan2013topology,lee2017quantifying}, and the availability of formulae for the cup-$i$ products allowed for the development of a theory of Persistence Steenrod modules accessing finer features of the data \cite{medina2018persistence}. In Quantum Field Theory, the BRST formalism \cites{BRS,Tyutin,KugoOjima} introduced cohomology to deal with gauge symmetries, and was further developed in the BV-formalism \cite{BV}. Master equations, introduced there, are also needed for the construction of viable actions in String Field Theory \cite{Zwiebach}, where fundamental chains are involved. The chain level viewpoint for the study of homology operations was brought to the forefront by String Topology \cite{Sullivanoverview}, and as served as a major driver of innovation \cites{TZ, hoch2}. The cubical viewpoint appears in physics naturally in the theory of renomalization \cite{KreimerCubical}; while Steenrod cup-$i$ products are utilized in the construction of effective actions of gapped systems \cite{kapustin2017fermionic}. This connection with physics motivated the development of effective versions of spin bordism \cites{brumfiel2016pontrjagin, brumfiel2018pontrjagin}, which prominently feature Steenrod cup-$i$ products and Cartan coboundaries.

As in \cite{may70generalapproach}, we take a more general approach to Steenrod operations that also includes Araki-Kudo-Dyer-Lashof operations on the mod-$p$ homology of infinite loop spaces \cites{araki56squaring, dyer62lashof}. We use the language of operads \cite{may72geometry}  to describe at the chain level the integral structure required to define (co)homology operations at every prime. We then describe effective constructions of this structure on three prominent models of the $E_\infty$-operad, identifying elements in them that represent Steenrod operations in the mod-$p$ homology of their algebras; these are the Barratt-Eccles \cite{berger04combinatorial}, surjection \cite{mcclure03cochain}, and $U(\mathcal M)$ \cite{medina2020prop1} operads. Since the cochains of simplicial sets are equipped with effective and compatible algebra structures over each of these operads, we are able to explicitly describe canonical products generalizing the \mbox{cup-$i$} products of Steenrod to every prime. An alternative approach based on the Eilenberg-Zilber contraction can be found in \cite{gonzalez2005hpt}. Up to now, although its existence is warrantied by acyclic model argument \cite{eilenberg1953acyclic}, no $E_\infty$-structure was known on the cochains of cubical sets. Using an $E_\infty$-bialgebra structure on the chains of the interval we effectively construct an $U(\mathcal M)$-algebra structure on them, and as a consequence provide a description of Steenrod products for cochains of cubical sets as well. The $p=2$ part of our constructions specialize in the simplicial context to the definition of cup-$i$ products given by Steenrod \cite{steenrod47products} and recovered by several authors \cites{mcclure03cochain, berger04combinatorial, medina2018axiomatic}, and in the cubical setting to the cup-$i$ products of  Kadeishvili \cite{kadeishvili1998dg} and Kr\v{c}\'{a}l-Pilarczy \cite{pilarczyk2016cubical}.

We also contribute to a growing body of open source software \cites{manero2020effective, pilarczyk2015contraction, sagemath} resulting from an effort to rewrite key concepts from Homotopy Theory in effective forms. Uses in Applied Topology are motivating some of this work \cite{tauzin2020giottotda}, but the impetus also comes from Voevodsky's program for a comprehensive, computational foundation of mathematics based on the homotopical interpretation of type theory \cite{bauer2017hott}. A \texttt{Python} implementation of all the constructions in this article can be found in A.M-M.'s website.\footnote{Currently hosted at \url{https://www.medina-mardones.com/}}

An outline of the article is described next.
We first introduce, in Section~2, the conventions we will follow regarding chain complexes, simplicial sets and cubical sets.
Then, in Section~3, we review the key notions from group homology we will use for cyclic and symmetric groups. Section~4 is devoted to the language of operads and related structures that we use, in Section~5, to define an integral structure at the chain level inducing Steenrod operations for every prime. Section~6, the bulk of this work, presents effective constructions of Steenrod-Adem structures on the Barratt-Eccles, surjection, and $U(\mathcal M)$ operads.
It also describes a natural $U(\mathcal M)$-algebra structure on the cochains of simplicial and cubical sets inducing natural Steenrod-Adem structures on them.
We end, in Section~7, with an overview of some connections of this work to important geometric and combinatorial structures, and provide an outline of future research directions.

\section*{Acknowledgement}

The authors thank Dennis Sullivan, Kathryn Hess, Clemens Berger, John Morgan, Greg Brumfiel, Dev Sinha, Jens Kjaer, Calista Bernard, and Andy Putnam, for insightful discussions and valuable comments, and to the anonymous referee for many helpful suggestions.