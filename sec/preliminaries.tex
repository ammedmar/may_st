
\section{Preliminaries} \label{s:preliminaries}

\subsection{Chain complexes} \label{s s:chain cpx}

Let $R$ be a ring.
We denote by $(\mathbf{Ch}_R, \otimes, R)$ the symmetric monoidal category of homologically graded chain complexes of $R$-modules.
The set of $R$-linear maps between chain complexes as well as the tensor product of chain complexes are regarded as chain complexes in the usual way:
\begin{equation*}
\Hom(A, A^\prime)_n = \big\{f \ |\ a \in A_m \Rightarrow f(a) \in A^\prime_{m+n} \big\}, \qquad
\partial f = \partial \circ f - (-1)^{|f|}f \circ \partial,
\end{equation*}
\begin{equation*}
(A \otimes A^\prime)_n = \bigoplus_{p + q = n} A_p \otimes A^\prime_q, \qquad \partial (a \otimes a^\prime) = \partial a \otimes a^\prime + (-1)^{|a|} a \otimes \partial a^\prime.
\end{equation*}
We embed the category of $R$-modules as the full subcategory of $\mathbf{Ch}_R$ with objects concentrated in degree $0$.
The endofunctor $\Hom(-, R)$ is referred to as \textit{linear duality}.
We notice that if a chain complex is concentrated in non-negative degrees then its linear dual concentrates on non-positive ones.

The rings we will mostly be interested in are the group rings $\Z[\mathrm G]$ and $\mathbb{F}_p[\mathrm G]$ of finite groups, where $p$ is prime and $\mathbb{F}_p$ is the field with $p$ elements.

\subsection{Simplicial sets}

The \textit{simplex category} $\triangle$ is defined to have an object $[n] = \{0, \dots, n\}$ for every $n \in \mathbb{N}$ and a morphism $[m] \to [n]$ for each order-preserving function from $[m]$ to $[n]$.
The morphisms $\delta_i \colon [n-1] \to [n]$ and $\sigma_i \colon [n+1] \to [n]$ defined for $0 \leq i \leq n$ by
\begin{equation*}
\delta_i(k) =
\begin{cases} k & k < i, \\ k+1 & i \leq k, \end{cases}
\quad \text{ and } \quad
\sigma_i(k) =
\begin{cases} k & k \leq i, \\ k-1 & i < k, \end{cases}
\end{equation*}
generate all morphisms in the simplex category.

A \textit{simplicial set} $X$ is a contravariant functor from the simplex category to the category of sets, and a simplicial map is a natural transformation between two simplicial sets.
As is customary, we use the notation
\begin{equation*}
X\big( [n] \big) = X_n, \qquad X(\delta_i) = d_i, \qquad X(\sigma_i) = s_i,
\end{equation*}
and refer to elements in the image of any $s_i$ as \textit{degenerate}.

For each $n \in \mathbb{N}$, the simplicial set $\triangle^n$ is defined by
\begin{equation*}
\triangle^n_k = \Hom_{\triangle} \big( [k], [n] \big), \qquad
d_i(x) = x \circ \delta_i, \qquad
s_i(x) = x \circ \sigma_i,
\end{equation*}
and any simplicial set can be expressed as a colimit of these
\begin{equation*}
X \cong \colim_{\triangle^n \to X} \triangle^n.
\end{equation*}
We represent the non-degenerate elements of $\triangle^n_k$ as an increasing sequence $[v_0, \dots, v_k]$ of non-negative integers each less than or equal to $n$.

The functor $N_\bullet$ of \textit{normalized chains} (with $R$-coefficients) is defined as follows:
\begin{equation*}
N_\bullet(X; R)_n = \frac{R \{ X_n \}}{R \{ s(X_{n-1}) \}}
\end{equation*}
where $s(X_{n-1}) = \bigcup_{i=0}^{n-1} s_i(X_{n-1})$, and $\partial_n \colon N_\bullet(X)_n \to N_\bullet(X)_{n-1}$ is given by
\begin{equation*}
\partial_n = \sum_{i=0}^{n} (-1)^id_{i}.
\end{equation*}
The functor of \textit{normalized cochains} $N^\bullet$ is defined by composing $N_\bullet$ with the linear duality functor $\Hom(-, R)$.

It is convenient to emphasize that
\begin{equation*}
N_\bullet(X; R) = \colim_{\triangle^n \to X} N_\bullet(\triangle^n; R).
\end{equation*}

\subsection{Cubical sets}

The \textit{cube category} $\square$ is the free strict monoidal category with a \textit{bipointed object}
\begin{equation*}
\begin{tikzcd}
1 \arrow[r, bend left, "\delta^0"] \arrow[r, bend right, "\delta^1"'] & 2 \arrow[r, "\sigma"] & 1
\end{tikzcd}
\end{equation*}
such that $\sigma \circ \delta^0 = \sigma \circ \delta^1 = \mathrm{id}$.
Explicitly, it contains an object $2^n$ for each non-negative integer $n$ and its morphisms are generated by the \textit{coface} and \textit{codegeneracy maps} defined by
\begin{align*}
\delta_i^\varepsilon & = \mathrm{id}_{2^{i-1}} \times \delta^\varepsilon \times \mathrm{id}_{2^{n-1-i}} \colon 2^{n-1} \to 2^n, \\
\sigma_i & = \mathrm{id}_{2^{i-1}} \times \, \sigma \times \mathrm{id}_{2^{n-i}} \colon 2^{n} \to 2^{n-1}.
\end{align*}

A \textit{cubical set} $X$ is a contravariant functor from the cube category to the category of, sets and a cubical map is a natural transformation between two cubical sets.
As is customary, we use the notation
\begin{equation*}
X\big( 2^n \big) = X_n \qquad X(\delta^\varepsilon_i) = d^\varepsilon_i \qquad X(\sigma_i) = s_i,
\end{equation*}
and refer to elements in the image of any $s_i$ as \textit{degenerate}.

For each $n \in \mathbb{N}$, the cubical set $\square^n$ is defined by
\begin{equation*}
\square^n_k = \Hom_{\square} \big( 2^k, 2^n \big), \qquad
d^\varepsilon_i(x) = x \circ \delta^\varepsilon_i, \qquad
s_i(x) = x \circ \sigma_i.
\end{equation*}
We represent the non-degenerate elements of $\square^n$ as sequences $x_1 \cdots\, x_n$ with each $x_i \in \big\{[0], [1], [0,1]\big\}$.
For example, $[0][01][1]$ represents $\delta^1 \times \mathrm{id} \times \delta^0$.
Any cubical set can be expressed as a colimit of these
\begin{equation*}
X \cong \colim_{\square^n \to X} \square^n.
\end{equation*}

The functor $N_\bullet$ of \textit{normalized chains} (with $R$-coefficients) is defined as follows: The chain complex $N_\bullet(\square^1)$ is simply the cellular chain complex of the interval,
isomorphic to
\begin{equation*}
\begin{tikzcd} [column sep = small, row sep = 0.1pt]
R\{[0], [1]\} & \arrow[l] R\{[0,1]\} \\
{[1] - [0]} & \arrow[l, |->] \left[0,1\right].
\end{tikzcd}
\end{equation*}
Set
\begin{equation*}
N_\bullet(\square^n; R) = N_\bullet(\square^1; R)^{\otimes n}
\end{equation*}
and define
\begin{equation*}
N_\bullet(X; R) = \colim_{\square^n \to X} N_\bullet(\square^n; R).
\end{equation*}

The functor of \textit{normalized cochains} $N^\bullet$ is defined by composing $N_\bullet$ with the linear duality functor $\Hom(-, R)$.