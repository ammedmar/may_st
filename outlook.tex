
\section{Outlook} \label{s:outlook}

This article looked at Steenrod operations from an algebraic viewpoint, a subject with rich geometric and combinatorial components as well.
For example, \cite{Postnikov} and \cite{medina2018prop2} independently introduced equivalent geometric representations of the cup-$i$ products in terms of stabilized arc surfaces \cite{KLP} and weighted ribbon graphs respectively.
In fact, an entire $E_\infty$-operad (prop) is constructed geometrically in this way.
We can also interpret the May--Steenrod structure in $U(\mathcal M)$ from the ``oriented surface" perspective, where the norm map $N$ of cyclic groups --~a key ingredient in Definition \ref{def: minimal cyclic resolution}~-- was identified in \cite{KLP} with a Dehn twist operator in connection with Connes' cyclic complex.

In Higher Category Theory, the paper \cite{medina2020globular} constructs a functor producing strict \mbox{$\infty$-cat}egories from group-like cup-$i$ coalgebras in a manner similar to \cite{steiner2004omega}.
In particular, the cup-$i$ constructions described in this article for standard simplices and cubes define, respectively, the Street nerve and cubical nerve of strict $\infty$-categories.
We anticipate that the more general Steenrod $(p,i)$-products constructed in this work will also have deep combinatorial interpretations.

There is a functorial approach to the theory using the formalism of Feynman categories \cite{feynman},
which renders all the structures and notions natural.
This includes cyclic, planar cyclic as well as Berger's pre-operads \cite{BergerRecog}.
Their interplay is of independent interest \cite{BergerKaufmann, feyrep} and will be linked directly to the constructions of this paper.

In this article we have not focused on the operations that exist non-trivially for \mbox{$E_n$-algebras} with $n$ finite, see part III of \cite{may76homology}.
A treatment close to ours for the $E_2$ case was given in \cite{Tourtchine}.
Effective constructions for these Dyer--Lashof--Cohen operations should be related to the geometry and combinatorial structure of configuration spaces \cite{KZhang, sinha2013littledisks, berger04combinatorial, ayala2014configuration} and higher categories \cite{Bathigher, BalFiedSchwVogt, Rezkhigher}, and will appear elsewhere.

