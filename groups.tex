
\section{Group homology} \label{s:goup homology}

Fixing notation, let $\mathrm{S}_n$ be the symmetric group of $n$ elements and $\mathrm{C}_n$ be the cyclic group of order $n$ thought of as the subgroup of $\mathrm{S}_n$ generated by an element~$\rho$. We denote this inclusion by $\iota : \mathrm C_r \to \mathrm S_r$.

A \textit{resolution} in $\mathbf{Ch}_R$ is a quasi-isomorphism $P \to M$ with each $P_n$ a free \mbox{$R$-module}. We will use the fact, explained in Section 6.5 of \cite{jacobson1989algebra}, that such resolutions exist for any chain complex $M$ concentrated in non-negatively degrees, in particular, for the monoidal unit $R$.

Let $\mathrm G$ be a group and $M$ an $R[\mathrm G]$-module, we define the \textit{homology of $\mathrm G$ with coefficients in $M$}, denoted $H(\mathrm G; M)$, as the homology of any chain complex $P \otimes_{R[\mathrm G]} M$ with $P \to R$ a resolution. We will be particularly interested in the case when $M = \mathbb F_p(q)$ is the trivial or sign $\mathbb F_p[\mathrm{S}_n]$-module depending on the parity of~$q$.

We now review the group homology of finite cyclic groups. For any ring $R$ the elements
\begin{equation} \label{eq: T and R definition}
\begin{split}
T &= \rho - 1, \\
N &= 1 + \rho + \cdots + \rho^{n-1},
\end{split}
\end{equation}
in $R[\mathrm{C}_n]$ generate the ideal of annihilators of each other. Therefore, the chain complex $\mathcal W(n)$ concentrated in non-negative degrees with $\mathcal W(n)_d$ the free $\Z[\mathrm{C}_n]$-module generated by one element $e_d$, and differential
\begin{equation*}
\partial(e_d) = \begin{cases}
Te_{d-1} & d \text{ odd,} \\
Ne_{d-1} & d \text{ even,}
\end{cases}
\end{equation*}
defines a resolution $\mathcal W(n) \to R$ in $\mathbf{Ch}_{R[\mathrm{C}_n]}$. It follow from a straightforward computation that for any prime $p$ and integer $q$
\begin{equation*}
H_i(\mathrm{C}_p; \mathbb{F}_p(q)) = \mathbb{F}_p.
\end{equation*}

The homology of $\mathrm{S}_n$ is harder to compute. With untwisted coefficients, the method of computation followed by several authors was to prove the injectivity of this homology into that of the infinite symmetric group and take advantage of a natural Hopf algebra structure on it. A powerful result stemming from the deep connection of this question with infinite loop space theory, is the existence of a homology isomorphism of spaces
\begin{equation*}
\Z \times B\mathrm{S}_\infty \to Q(S^0) = \Omega^\infty \Sigma^\infty (S^0)
\end{equation*}
credited to Dyer-Lashof \cite{dyer62lashof}, Barratt-Priddy and Quillen \cite{barratt1972priddy}.

In this work, we are interested in the mod-$p$ homology of $\mathrm{S}_p$ ($p$ a prime) which, as explained in \cite{adem2004milgram} Corollary VI.1.4, is detected by the inclusion $\iota : \mathrm{C}_p \to \mathrm{S}_p$ whose homological kernel is described next.

\begin{lemma} \label{lem: Thom's theorem}
	Let $p$ be an odd prime and $q$ an integer. Consider
	\begin{equation*}
	(\iota_\ast)_n : H_n(\mathrm{C}_p; \mathbb{F}_p(q)) \to H_n(\mathrm{S}_p; \mathbb{F}_p(q))
	\end{equation*}
	then
	\begin{enumerate}
		\item If $q$ is even, $(\iota_\ast)_n = 0$ unless $n = 2t(p-1)$ or $n = 2t(p-1) - 1$,
		\item If $q$ is odd, $(\iota_\ast)_n = 0$ unless $n = (2t+1)(p-1)$ or $n = (2t+1)(p-1)-1$,
	\end{enumerate}
	for some integer $t$.
\end{lemma}

\begin{proof}
	This is proven as Theorem 4.1 in \cite{steenrod53cyclic} where Thom is also credited with a different proof.
\end{proof}

In section 5 we will see how the cohomology of symmetric groups defines operations on algebras that are commutative up to coherent homotopies. Preparing for that, we first developed the language of $\Gamma$-modules and operads, where the integral structure defining Steenrod operations is clearly expressed.